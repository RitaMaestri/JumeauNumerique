\documentclass{article}
\begin{document}
	\title{Ground controller tasks and tools}
	\maketitle


\section{Environment and geographical splitting}

There are three control towers where the ground controller can work: southern, northern and central. The central tower is used at night, from 10 pm to 6:30 am, and at most 3 ground controllers can work there at the same time. The southern and northern towers are used during day time, and at most 2 ground controllers can work at the same time in each of the two tower. So, during daytime there can be at most 4 ground controllers working at the same time.

During daytime, there are 4 PAI (point d'arret intermediaire) between the northern ground and the southern ground: Middle1 (taxiway N), Middle2 (F), Middle3 (B) and Middle4 (Q). Those points are the borders between north sector and south sector.

Some tasks are different between the north and and south. For example, GCs on the north part have to handle with push-backs when there are some specific apron controllers who do that for the south part of the field (in fact from 10pm to 7am, there is no apron controller and the GCs have to handle it also on the south). 

In the northern (southern) tower, there is a ground controller always working, and a backup ground controller that begins to work whenever the first GC choose to split his sector in two. Each sub-sector is covered by one specific radio frequency. When the ground controller works on the full northern(southern) sector, he or she merges the two radio frequencies.

In same room of the ground controllers, we can also find the delivery controller and the runway controller workstations. The apron controller and the tower supervisor work in separate rooms.

\section{Tools}
In the ground controller workstation there are several tools:
\begin{itemize}
	\item \textbf{Screen.} It contains an interactive map of CDG airport platform (it is possible to zoom in and out). The GC usually chooses to zoom on its sector, but to keep an open window on other sectors to be able to anticipate if an aircraft is directed towards his sector.	
	On the map, the position of each aircraft on the platform is constantly updated. To each aircraft position is associated its identification code, speed and aircraft type. 
	\item \textbf{View.} The information about aircraft position is gained by direct sight, too.
	\item \textbf{Paper strip.} In front of the GC there is a wooden holder that can host 20 paper strips. Each paper strip corresponds to an aircraft that the ground controller has to handle. Some important information is already written on the strip when it's printed, and some other information is added by the GC.
	In particular, when it's printed we can find: the aircraft ID, the type of departure, runway of departure, stand number. Furthermore, arriving aircraft and departing aircraft can be distinguished by the color of the strip (blue for departures, white for arrivals). 
	The GC will add: the proposed holding point at the runway, the path.
	
	The ground controller places the strips on the wooden carrier in positions that help him remind the priorities of the various aircraft.
	
	The strips are printed 10-15 minutes before the aircraft lands in case of arrivals, as soon as the delivery controller gives the permission for the startup, usually 10 minutes before the startup, in case of departure.
	
	\item\textbf{Sky radar screen.} It shows the position of aircraft that are flying around the airport.
	\item\textbf{Frequency regulator.} It allows to transmit, monitor and group frequencies.
	\item\textbf{Radio.} It is used for communications with the pilots.
	\item \textbf{Phone.} It is used for communications with the controllers that are not in the same tower.
	

\end{itemize}

\section{Effects of ground controller decisions on the traffic}
\subsection{Frequency splitting}
The splitting of the work is an decision which has some important effects on the taxi time. When the GC choose to split the area, the communication between him and the beackup GC becomes essential because he has to give instructions about the current state of traffic. But at the same time, he has to give instructions to planes he is in charge of, listen to them, think about his strategy, and speak to other controllers. Moreover, between two sections, planes have to stop to ensure the security.
\subsection{Shift of GC in the workstation}
Another event that can cause a lengthening of the taxitime is the arrival of a new GC. The current GC has to give information about the traffic, the weather, the tools' condition and, at the same time, continue following the planes.

Moreover, according to the time available for the current and the new GC, the change can be quite brutal. In high workload situation it can difficult to handle. 

\subsection{Frequency change}
The first reason why an aircraft slows down or stops is because he's not receiving orders from the ground control. This happens especially when the pilot has to change frequency when he enters a new sector (from east to west or from north to south and vice versa): if the new GC isn't quick enough in giving the clearance to the new aircraft, the pilot will stop at the PAI.
How does the switch between frequencies happens? The first ground controller, that has taken care of the aircraft until the PAI, tells the pilot to switch the frequency at that point, to be able to speak with the new ground controller. It can happen that the GC does not warn the pilot in time, and the pilot crosses the PAI and continues taxiing without changing the frequency also for 500 m, until he's told to change. This happens once or several times a day, and it is a safety issue. \textbf{The GC in this situation would need the pilots help to stop sometimes}.

\subsection{Clearance change}
The aircraft may face other aircraft while he uses the assigned path. In this case, the ground controller has to give the priority to one of the two based on the slot, type of aircraft.

\subsection{Apron Area Management}
Especially in terminal 1, which is the oldest and less efficient terminal in terms of traffic, you could have issues of priority between arriving and departing aircraft. A case that often happens is that an aircraft cannot enter the taxiway A in terminal 1 because another aircraft is pushing back, occupying the taxiway. In this case, one of the two has to stop (pushing back or entering the apron area). The same happens in terminal 3, where there is only one way for entering and exiting. 

\section{Communication with other agents}
\subsection{Ground controllers}
In order to coordinate the work of ground controllers, they use to exchange a lot of information. They can do it directly or by using their phones if it is easier. They are exchanging information about current planes, potential sources of traffic's complexity or incoming strategy.

\subsection{Runway controller}
\textbf{Arriving aircraft}: the runway controller tells the pilot to set the frequency on Ground Control once the aircraft has crossed the departure runway (in fact, because of the configuration of runways, an arriving aircraft has always to cross the departure runway to get to the stand). 
He is in charge of deciding the exit taxiway from the runway.
Non-common holding point and coordination on strategies.

\textbf{Departing aircraft}: in standard situations, the GC just passes the paper strip containing all information about the aircraft to the runway controller when it's the right time.
The runway controller can change the holding point decided by the ground controller if he says it early enough. He can ask for a change of holding point up to the moment when it's not possible for the ground controller to warn the pilot in time, so also 2 minutes before the aircraft gets to the holding point. But he usually does it when the aircraft is far away, for example when there is an aircraft stuck at the holding point.

\subsection{Delivery controller}
The GC is contacted by a pilot asking for a push-back ealier than the TSAT. He or she calls the delivery controller to check if is ok to push-back. Or the GC is contacted by a pilot who has a problem and would not be able to start a the planed hour, so he or she calls the delivery controller to ask for the plane to be pull back in the DMAN.

\subsection{Apron controller}
Use of non-standard procedures and coordination on strategies and priorities. The more planes the GC handles, the more it happens

\subsection{IFR room}
This interaction does not happen often, and it usually passes through the supervisor. A GC can ask to the IFR room to reroute an aircraft or make him wait on the sky if he has some serious traffic issues on the ground. Rorward information given by pilots

\subsection{Ground handling services}
For example for asking for a tug if the GC has time to do it.

\subsection{Pilots}
Pilots can not do something without the clearance gived by the GC. 
They are talking together by radio, using the frequency that covers this geographical part of the field. Between each section pilots have to change the frequency they are using and wait to get in touch with the new GC before carrying on their taxiing. 

The GC can give a total or partial clearance to pilots. In particular, we can distinguish different kinds of messages based on if the aircraft is arriving or departing.

\textbf{Arrivals}: the runway controller gives the clearance to land or to stay in the sky and the clearance to cross the inner runway or to stop at the waiting point. When the pilot is about to finish the crossing calls the ground controller(Ex.pilot:"\textit{AFR123. We are crossing runway 27L at D6.}). There are several possibilities here:
\begin{itemize}
	\item the GC doesn't answer because he's too busy handling the other aircraft. In this case the aircraft stops.
	\item the GC answers but doesn't have time to give the clearance, so he just says that he'll call back. In this case the aircraft stops (Ex. GC: ``\textit{AFR123, I'll call you back}").
	\item the GC gives the clearance only about the first move (ex. turn left) and he calls back later (Ex. GC: ``\textit{AFR123, turn left and I'll call you back}").
	\item There are not many aircraft in the sector handled by the GC, so he will give the whole clearance, until the stand or until Middle1(2,3,4) (Ex. GC: ``\textit{AFR123, turn right on D and left on F, stop at Middle2}").
	\item There are many aircraft in the sector so the GC is not able to give the whole clearance, but only a partial one. The partial clearance includes the instruction to hold at a certain point in the taxiway, or to give priority to another aircraft (Ex. GC: ``\textit{AFR123, turn left on D and right on F, then hold short at B}" or ``\textit{AFR123, turn left on D then give way to AFR456 from your right}").
\end{itemize}

\textbf{Departures}: the GC has to handle the aircraft from the pushback, if there are no apron services in his sectors. If there is the apron service, the GC receives the aircraft at the PAI. 
In the first case, the GC gives the permission to push back (Ex. GC: ``\textit{AFR123, push back is approved facing west, call me back for taxi}" or ``\textit{AFR123, give way to the traffic rolling behind you and then push back is approved, call me back for taxi}" or ``\textit{AFR123, push back approved, could you make a long one to give way to AFR456 to roll ahead of you}"). 

Then, the aircraft is ready for taxi. At this point the GC has to give the clearance. The clearance arrives until the last point included in the GC sector (which could be the holding point or Middle1(2,3,4))). The clearance could be total or partial, as in the arrival case.

\subsection{Paths's strategies}
Most of the time, the path chosen by GCs is linked to where the aircraft is coming from (the \textit{PAI} which is the way the aircraft is leaving the terminal) and which runway it is going to take-off from. It's normal that according to one \textit{PAI} you find many paths because first, the path will depend on which runway the aircraft is going to take-off from. For example, if you take the \textit{PAI A}, if the aircraft has to roll to the northern runway, GCs already have different paths regarding the configuration. If you take into consideration only the departures from the \textit{PAI A} to the runways \textit{27L} you'll see the path is different according to does the aircraft leaves the Terminal 1 via the taxiway \textit{NA1}, which is the one linked to the Terminal 1's east bound, or via the taxiways \textit{B} or \textit{D}.
From a given \textit{PAI} to a given runway, you have several potential paths but you don't have thousand. From the \textit{PAI NA1} to \textit{27L}, the only choice GCs have is taxiing via the taxiway \textit{N}, then via the taxiways \textit{Q} or \textit{B} and finally GCs are going to choose the holding point \textit{Q1}, \textit{Q2}, \textit{Q3}, \textit{Q4}, \textit{Q5} or \textit{Q6}. The difference in the path used could first be the holding point where they have to leave the aircraft to, for example if GCs have to go to holding points \textit{Q1}, \textit{Q2}, \textit{Q3}, they rather use the taxiway \textit{Q} and to split traffic, they are going to use the taxiway \textit{B} to reach holding points \textit{Q4}, \textit{Q5} and \textit{Q6}. For the southern part of the field, from any \textit{PAI} to any runway, GCs have 10 to 20 paths they can use because planes can turn left or right at almost every crossroads. The reasons for a non-standard path could be:
\begin{itemize}
    \item an unusable taxiway (closed or work in progress or an aircraft stuck in it);
    \item holding points strategies;
    \item priority strategies;
    \item splitting sectors strategies;
\end{itemize}
Standard procedures makes a traffic flow direction standard but taxiways can be used in opposite direction. This needs coordination between GCs and it happens all the time.


\section{Pilots}
Pilots have a GPS with the map of the taxiways, but they have to know the names of the taxiways to be able to interact with the GC. The pilots also know their stand, so they are able to predict in some way the taxiways that they will have to use. Anyway, they must be able to adapt to what the ground controller tells them, and this will happen more or less easily based on the familiarity that the pilot has with the airport.

The speed of taxiing is influenced by the knowledge that the pilot has of the airport.

\section{Tower supervisor}
He or she is in charge of the coordination with the whole world: weather conditions, emergencies, strategies, configuration changes, delayed or early aircrafts, deicing bay, DMAN...

\section{Complexity of the traffic}
There are a lot of elements that bring complexity and these elements depend on the situation. The first variable that can change the definition of complexity is the number of sectors the GC handles and can be divided in three possibilities: night time (all the sectors merged), day time in a nominal traffic (north or south par of the field) and peak hour (only a quarter of the field). Then, there are also elements of complexity attached to the sector itself (for example if a GC is in charge of the south-east part of the field and they are facing east, he or she will have to deal with almost only arrivals and most of the time would be able to stick to the standard procedure). If we want to make a list of these elements, these are the main things to think about:
\begin{itemize}
    \item Number of planes;
    \item Emergencies: fire, medical emergency, mechanic emergency... ; 
    \item Weather conditions: low visibility\footnote{Pilots are going to go slowly because they are afraid and GCs have to change the way they give clearances (from "Give way to this aircraft" to "hold short in this place").}, storms... ;
    \item Change of configuration;
    \item Deicing procedure;
    \item Closed taxiways;
    \item Pilots' errors (several times a day);
    \item Restricted area;
    \item Towed aircraft\footnote{between maintenance areas and terminals or between terminals}: they are taxiing slowly and most of the time opposite way of the main flow of traffic;
\end{itemize}
As soon as there are plenty of elements that determine if a situation is complex or not, it would be more interesting to ask to GCs how they deal with a complex situation, no matter what it is. For example, in a non-complex situation, GCs have more time to think about non-standard procedures in order to optimise their strategies and try to fit to pilots' demands. On the other hand, in a very complex situation, the standard procedures could also be unenforceable and GCs would have to create an "brand-new" response to this situation.\\
Finally, complexity can be measured by the number and importance of messages a GC has to deliver. The more messages a GC has to pass on, the less time is available to think about his or her strategy and so the more complex the situation becomes. At most, a GC can deliver around 6-7 messages\footnote{Pilots have to confirm that they received the message by repeating it so it doubles the length of a message.} per minute during a period a 10-15 minutes (it would be to difficult to carry on on that tempo during a longer period). 


\section{ABM}
Sources of variability in the work of GC that should be implemented in the ABM:
\begin{itemize}
	\item \textbf{Variability in the decision of splitting}. This decision is individual and can be take only by the GC. It could be see as a threshold, specific to each GC, of complexity and number of planes beyond which the GC asks for the splitting.
	\item \textbf{Effect of the splitting on the taxitime}. The slipping enables CGs to have to handle less planes but it also forces GCs to work together and it makes the ground controlling more complex.
	\item \textbf{Variability in the arrival of a new shift}. the shift can be more or less abrupt according to the GCs. The abruptness of the shift can degrade the communication and the work of the new GC and so affect the taxitime.
\end{itemize}


\end{document}


