\documentclass{article}
\begin{document}
	
	\title{Ideas for ABM and Machine Learning Model}
	\maketitle
	
\section*{Jaufré suggestions}
There are two possible directions our project can take.
\begin{enumerate}
	\item We can implement a program that helps manage the flux of airport ground traffic.
	
	 To this aim, a model of the entire airport can be built. Thanks to the tuning of some key parameters, the model allows to optimize the taxi time averaged on the whole ensamble of aircrafts traveling on the airport platform on a large period of time. 
	
	The key parameters that we want to tune are not trivial ones (such as aircraft speed during taxiing), nor structural ones (such as the construction of another terminal), but operational parameters. We want in fact to identify the optimal strategy for pilots and ground controller to adopt for optimizing the traffic flux.
	
	\item A program that forecasts the taxi time of the aircrafts crossing the airport case by case.
\end{enumerate}
A solution for each one of the two cases is proposed. For the first case, an agent based model, that allows to integrate structural and human/organizational factors in the same simulation. For the second case, a classical machine learning model, that predicts taxi time starting from the large amount of data collected during landing, take off and taxi procedures in the last few years.

\section{Agent Based Model}	
The Agent Based Model we drew up is described below. 

The aircraft platform is modeled as a network whose nodes represent intersections between the taxiways and the taxiways themselves are represented by links whose weight is the length of the road.

With additional datas (mainly those included in \textit{.che} files) we could model it more precisely by considering not only the length of the various roads but also the average speed of planes on them.

Modeling the taxiway in a realistic way is essential to obtain a realistic model, since the results of our previous data analysis show that the distance travelled by aircrafts from the parking area to the runway andv viceversa is one of the most influential variables for taxi time.

The agents of type \textit{aircraft} move along the taxiways, with a strategy decided partly by the agent of type \textit{ground controller}, that has a more general view of what's happening on the airport platform, and partly by the agent of type \textit{pilot}. Thanks to the tuning of pilot and ground controller strategies we expect to find the optimal one, and to compare it with the actual behavior of the pilots and the ground controllers to see if it fits or what can be improved.

Eventual delays caused by external factors such as weather, delay at the take off airport for incoming flights... can be introduced as random noise in the time of aircraft's movement.
\subsection{Required Data}
The data required to implement this model are the following.
\begin{itemize}
	\item Data about airport architecture: in particular data concerning taxiway and runways length.
	\item Data about the schedule of flights to be able to compute the total delay of the flights.
	\item All the data that correspond to the non-visual information available to ground controllers and to pilots and that guide them in building their movement strategy.
	\item Clarifications about standard procedures adopted by ground controllers and pilots.
	\item Data that can explain the eventual difference in strategy in front of the same situation, i.e. personal strategy of each pilot/ground controller when the standard procedure is not applicable.
\end{itemize}
\subsection{Collecting data}
For what it concerns the last type of data listed above, it is possible to collect information about personal strategies of pilots and ground controllers thanks to surveys conducted via phone calls.
\subsection{Added value of our project}
The great advantage of an agent based model of this kind is its refinement, since it includes a detailed simulation of the taxiway and of the procedures needed for an aircraft to land and take off.


Moreover, not only it identifies the causes of the eventual delay of the flights (the flights are late because the airport capacity is exceeded/ the speed of aircraft is too low/ the weather is bad...), but it also offers a solution from an operational and strategical point of view.

This can happen thanks to the integration in one only model of the structural (taxiway modeling as a network) and human/operational factors(strategy modeling).
\section{Classical Machine Learning Model}
A classical machine learning model can be used to forecast the taxi time of each aircraft based on airport data. 

\subsection{How machine learning works}
A machine learning algorithm whose aim is to predict a certain value $y_i$ (which is in our case the taxi time of an aircraft $i$) has to be fed with $n$ variables, called features, $x_{i,1},....x_{i, n}$, that $y_i$ is supposed to depend on. The advantage of a machine learning algorithm is that it infers the shape of the function $y_i(x_{i,1},....x_{i, n})$ autonomously, learning from the data. As many data (in our case, as many flights) are available to the machine learning algorithm, the more precise will be the prediction of $y$.

The dataset available to us contain a great quantity of data, so we can hope to have a pretty accurate prediction of the taxi time. 

\subsection{Advantages}
The advantage of building a machine learning algorithm for predicting taxi time is that it is relatively simple to develop once the right data are available, thanks to Python Machine Learning libraries. The most difficult part is to select the features (i.e. the data) that are the most suited for our problem.

\subsection{Limits}
The limit of an algorithm of this kind is the fact that the output of the algorithm is simply the taxi time, but the reasons of an eventual delay remain unknown.

\subsection{Required data}
A fundamental role in the choice of features is played by the situation in which the prediction will be used: we must feed our machine learning algorithm with data that are available at the time in which the prediction will be used. For example, if the predicted taxi time will be used by ground controllers one hour earlier than the predicted take off/landing, data such as approximate weather condition, type of aircraft, scheduled time of landing/take off can be used.
If the prediction is used five minutes earlier, more informations such as the taxi time of the previous aircrafts and the number of aircrafts on the platform can be added as features.

We can see from this example that the development of the program depends heavily on the use that will be made of the program itself.

For this reason it is important to agree with the ground controllers the situation in which they will use the algorithm and if it's useful for them to know only the taxi time and not the reasons why it is high or low. 
\end{document}